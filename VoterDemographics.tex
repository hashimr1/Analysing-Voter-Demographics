% Options for packages loaded elsewhere
\PassOptionsToPackage{unicode}{hyperref}
\PassOptionsToPackage{hyphens}{url}
%
\documentclass[
  9pt,
  ignorenonframetext,
]{beamer}
\usepackage{pgfpages}
\setbeamertemplate{caption}[numbered]
\setbeamertemplate{caption label separator}{: }
\setbeamercolor{caption name}{fg=normal text.fg}
\beamertemplatenavigationsymbolsempty
% Prevent slide breaks in the middle of a paragraph
\widowpenalties 1 10000
\raggedbottom
\setbeamertemplate{part page}{
  \centering
  \begin{beamercolorbox}[sep=16pt,center]{part title}
    \usebeamerfont{part title}\insertpart\par
  \end{beamercolorbox}
}
\setbeamertemplate{section page}{
  \centering
  \begin{beamercolorbox}[sep=12pt,center]{part title}
    \usebeamerfont{section title}\insertsection\par
  \end{beamercolorbox}
}
\setbeamertemplate{subsection page}{
  \centering
  \begin{beamercolorbox}[sep=8pt,center]{part title}
    \usebeamerfont{subsection title}\insertsubsection\par
  \end{beamercolorbox}
}
\AtBeginPart{
  \frame{\partpage}
}
\AtBeginSection{
  \ifbibliography
  \else
    \frame{\sectionpage}
  \fi
}
\AtBeginSubsection{
  \frame{\subsectionpage}
}
\usepackage{amsmath,amssymb}
\usepackage{lmodern}
\usepackage{ifxetex,ifluatex}
\ifnum 0\ifxetex 1\fi\ifluatex 1\fi=0 % if pdftex
  \usepackage[T1]{fontenc}
  \usepackage[utf8]{inputenc}
  \usepackage{textcomp} % provide euro and other symbols
\else % if luatex or xetex
  \usepackage{unicode-math}
  \defaultfontfeatures{Scale=MatchLowercase}
  \defaultfontfeatures[\rmfamily]{Ligatures=TeX,Scale=1}
\fi
\usetheme[]{Pittsburgh}
\usecolortheme{crane}
\usefonttheme{structurebold}
% Use upquote if available, for straight quotes in verbatim environments
\IfFileExists{upquote.sty}{\usepackage{upquote}}{}
\IfFileExists{microtype.sty}{% use microtype if available
  \usepackage[]{microtype}
  \UseMicrotypeSet[protrusion]{basicmath} % disable protrusion for tt fonts
}{}
\makeatletter
\@ifundefined{KOMAClassName}{% if non-KOMA class
  \IfFileExists{parskip.sty}{%
    \usepackage{parskip}
  }{% else
    \setlength{\parindent}{0pt}
    \setlength{\parskip}{6pt plus 2pt minus 1pt}}
}{% if KOMA class
  \KOMAoptions{parskip=half}}
\makeatother
\usepackage{xcolor}
\IfFileExists{xurl.sty}{\usepackage{xurl}}{} % add URL line breaks if available
\IfFileExists{bookmark.sty}{\usepackage{bookmark}}{\usepackage{hyperref}}
\hypersetup{
  pdftitle={STA130 Final Project},
  pdfauthor={Chetanya Saxena (1007320533), Maryam Ansari (1006917204), Raazia Hashim (1006819454) and Suchi Aidasani (1006753229) - Group 132},
  hidelinks,
  pdfcreator={LaTeX via pandoc}}
\urlstyle{same} % disable monospaced font for URLs
\newif\ifbibliography
\usepackage{graphicx}
\makeatletter
\def\maxwidth{\ifdim\Gin@nat@width>\linewidth\linewidth\else\Gin@nat@width\fi}
\def\maxheight{\ifdim\Gin@nat@height>\textheight\textheight\else\Gin@nat@height\fi}
\makeatother
% Scale images if necessary, so that they will not overflow the page
% margins by default, and it is still possible to overwrite the defaults
% using explicit options in \includegraphics[width, height, ...]{}
\setkeys{Gin}{width=\maxwidth,height=\maxheight,keepaspectratio}
% Set default figure placement to htbp
\makeatletter
\def\fps@figure{htbp}
\makeatother
\setlength{\emergencystretch}{3em} % prevent overfull lines
\providecommand{\tightlist}{%
  \setlength{\itemsep}{0pt}\setlength{\parskip}{0pt}}
\setcounter{secnumdepth}{-\maxdimen} % remove section numbering
\ifluatex
  \usepackage{selnolig}  % disable illegal ligatures
\fi

\title{STA130 Final Project}
\subtitle{Analysing Voter Demographics - Liberal Party}
\author{Chetanya Saxena (1007320533), Maryam Ansari (1006917204), Raazia
Hashim (1006819454) and Suchi Aidasani (1006753229) - Group 132}
\date{December 7, 2020}

\begin{document}
\frame{\titlepage}

\begin{frame}{Introduction}
\protect\hypertarget{introduction}{}
The questions we chose to cover in our final project provide insight to
the Liberal Party about their current standing in voter opinions. For
this, we use past data and manipulate it to make predictions. The
findings of this project will contain information about the gender split
among the past votes in favor of Liberal Party as well as if Liberal
Party is the first choice of voters. For the research questions entailed
in this presentation, we are using the data from the 2019 Online
Canadian Election survey. We have treated the data we have of 25850
(after removing missing values) as a sample and the population is all
the people politically participating in voting; all Canadians over the
age of 18.

\begin{block}{Data Summary}
\protect\hypertarget{data-summary}{}
Each of our research questions involve different variables from the
Election Survey results. To familiarize the audience with them, their
descriptions are attached as follows: - \(votechoice\): Which party is
your first choice to vote for? - \(gender\): Which gender does the voter
belong to? (e.g.~man, woman or other) - \(bornin\_canada\): Were you
born in Canada?

The methods for data wrangling that we will be using are filtering,
grouping and mutating variables.
\end{block}
\end{frame}

\begin{frame}{Research Question 1}
\protect\hypertarget{research-question-1}{}
\begin{block}{Introduction}
\protect\hypertarget{introduction-1}{}
\textbf{What percentage of votes should the Liberal party expect from
people born in Canada during the election at this point in time?}

The problem explored in this question is the proportion of people who
would vote for the ``Liberal Party'' as their first choice as opposed to
the other parties collectively. I have used our sample to try and
predict a confidence interval for the Liberal party so that they know
what percentage of vote they should be expecting.
\end{block}

\begin{block}{Statistical Methods}
\protect\hypertarget{statistical-methods}{}
The statistical method we will use to approach our question is
bootstrapping. Bootstrapping is a method used to estimate the sampling
distribution of the population of eligible voters in Canada. We draw
many bootstrap samples of the same size (n = 22146) from the sample that
we have (in this case 1000 bootstrap samples) with replacement which
allows our bootstrap sample to have duplicate values. Next, for each
bootstrap sample, we filtered the data by vote choice of Liberal and by
people born in canada. We are not creating new data, rather we are
exploring variability from the original sample to create a range of
plausible values for the difference in proportions of votes.

This is a confidence interval, we found it by taking the middle 95\% of
the bootstrap distribution and it tells us that if we repeated this
procedure many times, 95\% of those times, the proportion of people who
would vote for the Liberal Party would fall within the 95\% confidence
interval.
\end{block}
\end{frame}

\begin{frame}[fragile]
\begin{block}{Visualization}
\protect\hypertarget{visualization}{}
We will visualize the proportion of people who would vote for the
liberal party as their first choice by creating a histogram which will
allow us to observe the shape and distribution of the proportions.This
distribution shows that the values range from around 0.295 to 0.310, and
the center is around 0.3025. Some extremities can be observed but this
provides an overview of the number of people who would vote for the
liberal party.
\includegraphics{finalproject_files/figure-beamer/unnamed-chunk-5-1.pdf}
\end{block}

\begin{block}{95\% confidence level}
\protect\hypertarget{confidence-level}{}
\begin{verbatim}
##      2.5%     97.5% 
## 0.3295821 0.3413709
\end{verbatim}
\end{block}
\end{frame}

\begin{frame}
\begin{block}{Results}
\protect\hypertarget{results}{}
If we repeated this process many times, 95\% of those confidence
intervals would include the true proportion of people born in Canada who
would vote for the Liberal party. It is always good to know where you
stand and identify what needs to be improved, therefore this data is
useful for the liberal party. This provides a good starting point when
deciding the approach to this election in that the results of this data
provides a reason for the liberal party to modify their policies so that
they can appeal to immigrants and greater demographics so that they
could increase the proportions of people who would vote for the liberal
party as their first choice.
\end{block}

\begin{block}{Conclusion}
\protect\hypertarget{conclusion}{}
Moreover, since this is a proportion in comparison to all the parties
collectively, this data tells the Liberal party that they have an
unspoken lead in the election because approximately 30\% is a large
percentage and therefore they should strengthen their weak policies and
further strengthen their stronger policies and consider different
strategies to market themselves more to the general voting population.
\end{block}
\end{frame}

\begin{frame}{Research Question 2}
\protect\hypertarget{research-question-2}{}
\begin{block}{Introduction}
\protect\hypertarget{introduction-2}{}
\textbf{Is the proportion of male people who voted for the Liberal Party
50\%?}

This research question can help determine the gender split among the
votes, and whether a certain gender preferred the Liberal Party.

Our population for this specific research question would be the people
that filled out the survey, voted for the Liberal Party and are within
the ages of 18 and 99 inclusive.
\end{block}
\end{frame}

\begin{frame}
\begin{block}{Visualization}
\protect\hypertarget{visualization-1}{}
The data, for all the eligible votes whose first choice vote is for the
Liberal Party, is seen on the Bar plot below and is divided by gender.

\includegraphics{finalproject_files/figure-beamer/unnamed-chunk-7-1.pdf}

From this data we can already conclude that more women had Liberal Party
as their first choice for their vote, compared to men.
\end{block}
\end{frame}

\begin{frame}
\begin{block}{Visualization (contd.)}
\protect\hypertarget{visualization-contd.}{}
This bar plot was created in steps. First the data was filtered by
removing missing (or NA) values and then a new data set called
`liberalvotes' was created which stores all the data for the individuals
that had listed the Liberal Party as their first choice of vote. The
reason a bar plot was chosen is because the variable gender has 2
different categorical levels which makes this plot the most suitable.
Furthermore, bar plots are more easily interpreted by those who lack
statistical knowledge.
\end{block}

\begin{block}{Statistical Methods}
\protect\hypertarget{statistical-methods-1}{}
I can test our original research question by carrying out a one-sample
hypothesis test. Our hypotheses for the test are listed as follows with
\(H_0\) being the null hypothesis and \(H_1\) being the alternative
hypothesis, with \emph{p} being the proportion of men.
\[H_0: p_{male}=0.5\] \[H_1: p_{male} \neq 0.5\]
\end{block}
\end{frame}

\begin{frame}
\begin{block}{Hypothesis Test Stimulation}
\protect\hypertarget{hypothesis-test-stimulation}{}
\includegraphics{finalproject_files/figure-beamer/unnamed-chunk-8-1.pdf}
\end{block}
\end{frame}

\begin{frame}[fragile]
\begin{block}{Results}
\protect\hypertarget{results-1}{}
\begin{verbatim}
##      2.5%     97.5% 
## 0.4896578 0.5106766
\end{verbatim}

Above is the result from a 95\% confidence interval (it is used to
calculate how confident we are with our data). The results from this
calculation state that ``We are 95\% confident that between 49\% and
51.1\% of people that voted for the Liberal party are male''. A narrow
confidence interval means that there is less variability in our data and
may explain why there is a large gap between our vertical lines and
histogram.

\begin{verbatim}
## [1] 0
\end{verbatim}

Our histogram visualization is symmetrical, centered at 0.5 (mean
proportion) and values range between 0.49 and 0.52. We use our test
statistic and original proportion to find our p-value (the probability
of obtaining test results in the least extreme scenario, under the null
hypothesis). In this case, the p-value is 0.
\end{block}

\begin{block}{Conclusion}
\protect\hypertarget{conclusion-1}{}
Since our p-value 0, we have very strong evidence against the null
hypothesis that the prevalence of males among those whose first choice
vote was the Liberal Party, is 0.5.
\end{block}
\end{frame}

\begin{frame}{Research Question 3}
\protect\hypertarget{research-question-3}{}
\begin{block}{Introduction}
\protect\hypertarget{introduction-3}{}
\textbf{What is the range of plausible values of the difference in
median age of Canadians who would vote for the Liberal party verses the
National Democratic Party as their first choice?}

We will determine whether a Canadian who would vote for the Liberal
Party, is on average younger or older than a Canadian who would vote for
the National Democratic Party. This is an important question to answer,
to determine if voters of the two parties belong to a similar age
demographic.
\end{block}

\begin{block}{Data Summary}
\protect\hypertarget{data-summary-1}{}
To explore this question, we shall look at Canadians 18 years or older,
that is who are eligible to vote in the election and who report that
they will vote for the Liberal Party or the National Democratic Party as
their first choice, and we will filter our data to only contain this
subset of observations.
\end{block}
\end{frame}

\begin{frame}
\begin{block}{Visualization}
\protect\hypertarget{visualization-2}{}
\includegraphics{finalproject_files/figure-beamer/unnamed-chunk-12-1.pdf}

We will visualize the age of Liberal and the NDP voters by creating two
box plots, which will give us a way to compare the center and
distribution of ages for the two groups. The center, that is the median,
which is the middle values is seen by the position of the line in the
center of the boxes. We can see that median age of NDP voters is much
lower than the median of Liberal voters. The distribution of NDP voters
is very skewed to the right (that is there is a longer right tail),
meaning that most voters tend to be younger. The distribution of Liberal
voters is also slightly right skewed, but not as much as the other
group. Finally, we see there are a few outliers in the NPD voter ages,
where some voters are a bit older than the rest in the group.
\end{block}
\end{frame}

\begin{frame}[fragile]
\begin{block}{Statistical Method}
\protect\hypertarget{statistical-method}{}
The statistical method we will be using to answer our question is
bootstrapping. Bootstrapping is a method used to estimate the sampling
distribution of the population of eligible voters in Canada. We draw
many bootstrap samples of the same size (n = 200) from the sample that
we have (in this case 5000 bootstrap samples) with replacement. Meaning
that our bootstrap samples may have duplicate values in them. Next, for
each bootstrap sample, we filtered the data by vote choice, calculated
the median age and found the difference between the median age of
Liberal voters and the median age of NDP voters.

Doing this, we are not creating new data, rather we are exploring
variability from the original sample to create a range of plausible
values for the difference in median age. This is a confidence interval,
we found it by taking the middle 80\% of the bootstrap distribution
(wider and narrower intervals could have been taken). This confidence
interval tells us that if we repeated this procedure many times, 80\% of
those times, the true difference in median age would fall inside the
confidence interval.

\begin{verbatim}
## 10% 90% 
##   6  16
\end{verbatim}
\end{block}
\end{frame}

\begin{frame}
\begin{block}{Results}
\protect\hypertarget{results-2}{}
The confidence interval we found through our bootstrap sampling for the
difference in median age for Liberal and NDP voters in between 6 and 16.
Since the interval is positive, and recall that we subtracted the median
age of Liberal voters from the median age of NDP voters, we can conclude
that on average, NDP voters tend to be younger than Liberal voters.
\end{block}

\begin{block}{Conclusions}
\protect\hypertarget{conclusions}{}
Since this is the case, the Liberal Party should be focusing their
campaigning efforts and new party policies towards a younger demographic
of Canadians, as this is something that the National Demographic Party
is succeeding in. The NDP is able to attract a younger voter base
through the fresh ideas they propose and the Liberal Party should be
taking a page out of their book. Finally, some limitations in answering
this question using the bootstrapping method is that the sample data
that we begin with may be biased and not fully representative of all
Canadian voters and since bootstrapping only reuses our sample data, the
interval we came up with may be biased.
\end{block}
\end{frame}

\begin{frame}{Limitations and Summary}
\protect\hypertarget{limitations-and-summary}{}
\begin{block}{Limitations}
\protect\hypertarget{limitations}{}
Some limitations to this data include the sample data may not be
completely representative of the population due to bias. In addition to
this, since samples are randomly drawn, this can lead to some
uncertainty. Another limitation that cannot be removed is that some of
the results might be biased due to confounding in variables.
\end{block}

\begin{block}{Summary}
\protect\hypertarget{summary}{}
The findings contained in this project serve to advise the Liberal Party
on how their campaign can be improved by targeting the right audiences.
They are as follows:

\begin{itemize}
\tightlist
\item
  The Liberal Party only has 30\% of the population sure about voting
  for them and they need to improve their campaign generally.
\item
  In general, the proportion of men to women in Liberal voters is not
  50\%. The Liberal Party needs to target their policies towards men.
\item
  They also need to target the upcoming new voters more, maybe by easier
  student loans or similar policies to earn the favor of this age group.
\end{itemize}
\end{block}
\end{frame}

\end{document}
